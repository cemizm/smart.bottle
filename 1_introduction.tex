\section{Introduction}
The consumption of water is crucial for human survival. In order to compensate the loss of water through normal activities, an average person has to consume about 2 to 4.5 litres of water per day under typical climatic conditions \cite{doi:10.1080/02508069608686494}. The long-term low intake of water can lead to various health problems. But managing water intake in everyday life can be a challenging task. The influence of time pressure, stress or in particular the mental state of elderly people, makes it difficult to remember to drink enough.

To support in such situations, there are several approaches using a microphone, to correlate swallowing sounds with the amount of water \cite{7031280,8229307}. While these approaches are producing results with a precision around 10 millilitres, they require the usage of a microphone around the neck. This reduces the user experience and increases the acceptance threshold to use such systems in a continuously way. But the latter one is crucial for accurate estimating of water intake on a daily basis. 

In this paper we examine various machine learning based approaches by using a 3-axis inertial measurement unit attached to the bottle, to estimate the water intake during the day. The base for these approaches are build by the labeled raw data of the accelerator and gyroscope sensors. To collect the learning data, an embedded development board consisting of a micro controller, a Bluetooth Low Energy module and an integrated inertial sensor is used. For the labeling process a mobile application is developed, which leverages the Bluetooth connection to the embedded device and manually records and labels the water intake events. In the next step the data is analyzed to find significant features for spotting water intake events.